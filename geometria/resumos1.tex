\documentclass[11pt,a4paper]{article}
\usepackage{amssymb}
\usepackage[ansinew]{inputenc}
\usepackage[brazil]{babel}

\pagestyle{empty}

\setlength{\topmargin}    {-9mm} 
\oddsidemargin = -5mm
\setlength{\textheight}{242mm}  
\setlength{\textwidth}{170mm}    
\linespread{1.3}
\setlength{\topmargin}    {-2mm}  
%
%
%
%%%%%%%%%%%%%%%%%%%%%%%%%%%%%%%%%%%%%%%%%%%%%%%%%%%%%%%%%%%%%%%%%%%%%%%%%%%%%%%%%%%%%%%%%%%%%%%%%%%%%%%%%%%%
%%%%%%%%%%%%%%%%%%%%%%%%%%%%% Voc� come�a a preencher a partir daqui %%%%%%%%%%%%%%%%%%%%%%%%%%%%%%%%%%%%%%%%%%%%%%%%%%
%%%%%%%%%%%%%%%%%%%%%%%%%%%%%%%%%%%%%%%%%%%%%%%%%%%%%%%%%%%%%%%%%%%%%%%%%%%%%%%%%%%%%%%%%%%%%%%%%%%%%%%%%%%%
%
%
\begin{document}
\title{\vspace{-1in}\parbox{\linewidth} \hfil\break\indent
{\sc MT 803D - T�picos em Matem�tica Aplicada \\
Geometria dos N�meros
}
\vspace{0.2cm}
\hrule
\vspace{1cm}} 


\date{} %A data deve ficar vazia!

\author{} \maketitle
\vspace{-2cm}
\large
{\noindent \bf Data: 21/11/2014 (Sexta-Feira)}\\
{\noindent \bf Local: Sala 224, IMECC}
\\[0.1\baselineskip]


\large
\noindent \textbf{14h:} Criptografia Baseada em Reticulados \\
\noindent Maiara Francine Bollauf - Unicamp
 \\[0.5\baselineskip]
\thispagestyle{empty}
\noindent{\bf Resumo:}
Todo sistema criptogr�fico est� baseado em problemas dif�ceis em Matem�tica. A estrutura de reticulados, por sua vez, nos traz dois problemas dif�ceis: problema do vetor mais curto (SVP) e o problema do vetor mais pr�ximo (CVP). Vamos apresentar ent�o esses problemas e estudar dois criptossistemas baseados em chaves p�blicas que est�o fundamentados na dificuldade de resolv�-los, que s�o o sistema GGH e o NTRU. Acredita-se que tais sistemas criptogr�ficos sejam eficientes na evolu��o dos computadores cl�sssicos e ainda, possam resistir � futura implementa��o do computador qu�ntico.
\\[1\baselineskip]
\large
\noindent {\textbf{15h15:} Provas alternativas para os Teoremas de Pick e de Ehrhart e Semigrupos Num�ricos}
Matheus Bernardini de Souza - Unicamp
\\[0.5\baselineskip]
\noindent{\bf Resumo:}
O Teorema de Pick relaciona a quantidade de pontos inteiros em um pol�gono convexo com a �rea desse poligono. No final do s�culo XIX, George Pick demonstrou esse resultado e em 2007 Murty e Thain usaram o Teorema de Minkowski para demonstrar o Teorema de Pick. 

O Teorema de Ehrhart nos d� uma forma para contagem de pontos inteiros em politopos (inteiros e racionais). Foi originalmente demonstrado na d�cada de 60 por Eug�ne Ehrhart e em 2009 Steven Sam deu uma nova demonstra��o para esse teorema usando o princ�pio da inclus�o-exclus�o, como principal ferramenta. 

Neste semin�rio, faremos as ideias usadas nas ``novas'' demonstra��es para os dois teoremas e daremos uma aplica��o do Teorema de Ehrhart em Semigrupos Num�ricos.


\end{document}
